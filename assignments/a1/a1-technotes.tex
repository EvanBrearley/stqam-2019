\documentclass[10pt,hidelinks]{article}
\usepackage[letterpaper, hmargin=0.75in, vmargin=0.75in]{geometry}
\usepackage{graphicx}
\usepackage[hyphens]{url}
\usepackage{hyperref}
\usepackage{listings}
\usepackage{pgf}
\usepackage{courier}

\parindent 0in
\parskip 1.5ex


\lstset{ %
language=Java,
basicstyle=\ttfamily\scriptsize,commentstyle=\scriptsize\itshape,showstringspaces=false,breaklines=true}


\begin{document}

\title{
SE465 \\
Software Testing, Quality Assurance, and Maintenance\\
Assignment 1 Technical Notes}
\author{Patrick Lam \\
{Release Date:  January 17, 2019} \\
}
\renewcommand{\today}{}
\maketitle


This document explains how to set up a working environment for
Assignment 1. I've chosen to use Vagrant plus glitch to make it easy to set up your
environment. I've tested these instructions on Debian GNU/Linux as
well as Windows. They should work on a Mac as well.

\section*{Initializing your virtual machine}

Install the following software:
\begin{itemize}
\item you should already have git, since you cloned the {\tt a1}
  repository;
\item virtualbox (\url{https://www.virtualbox.org/wiki/Downloads}); you don't need the extension pack or SDK;
\item vagrant (\url{https://www.vagrantup.com/downloads.html}).
\end{itemize}


You should have an {\tt se465-1191-USERNAME-a1} directory after cloning your provided git
repository. In that directory,
you will find a {\tt Vagrantfile}, {\tt bootstrap.sh},
along with what you need for Q1 and templates for your answers.

Next, you need to get vagrant to build your virtual machine.
\begin{itemize}
\item Go to the {\tt a1} subdirectory, and
  \begin{quote}
    \verb+$ vagrant up+
  \end{quote}
  This initializes your virtual machine and downloads the {\tt average} sample code into the virtual machine. \\
  potential pitfall: you may get a cryptic error about ``VT-x not available''. In that case, you need to go to your computer's BIOS settings and enable virtualization extensions. (See \url{http://superuser.com/questions/22915/how-do-i-enable-vt-x} for information.)
\item Start an {\tt ssh} seesion into the virtual machine you've just set up:
  \begin{quote}
    \verb+$ vagrant ssh+
  \end{quote} ~\\[-1em]
  potential pitfall: {\tt ssh} may not be set up/in the PATH on your (Windows) computer. Either put it there (git includes ssh), or ssh directly into your virtual machine:
  \begin{quote}
    \verb+> ssh vagrant@localhost -p 2222 -i <address-vagrant-ssh-tells-you>+
  \end{quote}
\end{itemize}
Great! Now you have a working virtual machine.

\section*{Editing files}
The Vagrant configuration is set up such that the {\tt shared}
directory is also visible inside the virtual machine in your home
directory there.  You can use your favourite text editor on your host
machine, or you can install vim or Emacs inside the VM and edit
there. Because the directory is shared, committing and pushing your
clone of the repository from your host machine will send us your
submission.

\end{document}


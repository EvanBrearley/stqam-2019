\documentclass[11pt]{article}
\usepackage{listings}
\usepackage{tikz}
\usepackage{enumerate}
\usepackage{url}
\usepackage{amssymb}
\usetikzlibrary{arrows,automata,shapes}

\lstset{basicstyle=\ttfamily \scriptsize,
  basicstyle=\ttfamily,
   columns=fullflexible,
   breaklines=true,
   numbers=left,
   numberstyle=\scriptsize,
   stepnumber=1,
   mathescape=false,
   tabsize=2,
   showstringspaces=false
}
\newtheorem{defn}{Definition}
\newtheorem{crit}{Criterion}

\newcommand{\handout}[5]{
  \noindent
  \begin{center}
  \framebox{
    \vbox{
      \hbox to 5.78in { {\bf Software Testing, Quality Assurance and Maintenance } \hfill #2 }
      \vspace{4mm}
      \hbox to 5.78in { {\Large \hfill #5  \hfill} }
      \vspace{2mm}
      \hbox to 5.78in { {\em #3 \hfill #4} }
    }
  }
  \end{center}
  \vspace*{4mm}
}

\newcommand{\lecture}[4]{\handout{#1}{#2}{#3}{#4}{Lecture #1}}
% 1-inch margins, from fullpage.sty by H.Partl, Version 2, Dec. 15, 1988.
\topmargin 0pt
\advance \topmargin by -\headheight
\advance \topmargin by -\headsep
\textheight 8.9in
\oddsidemargin 0pt
\evensidemargin \oddsidemargin
\marginparwidth 0.5in
\textwidth 6.5in

\parindent 0in
\parskip 1.5ex
%\renewcommand{\baselinestretch}{1.25}

% http://gurmeet.net/2008/09/20/latex-tips-n-tricks-for-conference-papers/
\newcommand{\squishlist}{
 \begin{list}{$\bullet$}
  { \setlength{\itemsep}{0pt}
     \setlength{\parsep}{3pt}
     \setlength{\topsep}{3pt}
     \setlength{\partopsep}{0pt}
     \setlength{\leftmargin}{1.5em}
     \setlength{\labelwidth}{1em}
     \setlength{\labelsep}{0.5em} } }
\newcommand{\squishlisttwo}{
 \begin{list}{$\bullet$}
  { \setlength{\itemsep}{0pt}
     \setlength{\parsep}{0pt}
    \setlength{\topsep}{0pt}
    \setlength{\partopsep}{0pt}
    \setlength{\leftmargin}{2em}
    \setlength{\labelwidth}{1.5em}
    \setlength{\labelsep}{0.5em} } }
\newcommand{\squishend}{
  \end{list}  }


\begin{document}

\lecture{21 --- February 27, 2017}{Winter 2017}{Patrick Lam}{version 1}

We'll now shift to the topic of bug-finding. This is generally useful to know
about, and is relevant to your course project.

A bug has got to be a violation of some specification. This might be
a language-level specification (e.g. don't dereference null pointers), or it
may be specific to an API that you are using.

For instance, you may have a method:
\begin{lstlisting}[numbers=none]
  // callers must acquire the lock
  static int reset_hardware(...) { ... }
\end{lstlisting}
\vspace*{-1em}
and then a caller in the Linux kernel:
\begin{lstlisting}[numbers=none]
  // linux/drivers/scsi/in2000.c:
  static int in2000_bus_reset(...) 
  {
    ...
    // no lock acquisition => a bug!
    reset_hardware(...);
    ...
  }
\end{lstlisting}

Examples of specifications:
\squishlist
\item {\tt len < 100};
\item {\tt x = 16*y + 4*z + 3};
\item {\tt fclose()} must be called after {\tt fopen()};
\item and example from above: callers of {\tt reset\_hardware()} must acquire lock.
\squishend

How do you get specifications? Well, programming languages come with some
specifications applicable to all programs in that language. Or, you can have
developers write specifications (an uphill battle). Or, you can automatically
infer specifications from source code, execution traces, comments, etc.
This includes static tools like Coverity, iComment, PR-Miner, etc. It also
includes dynamic tools like Daikon, DIDUCE, etc.

\paragraph{More on Coverity.}
Coverity is a commercial product
which can find many bugs in large (millions of lines) programs; it is
therefore a leading company in building bug detection tools.
We'll talk a bit more about Coverity in the future. Clients
(900+) include organizations such as Blackberry, Yahoo, Mozilla,
MySQL, McAfee, ECI Telecom, Samsung, Siemens, Synopsys, NetApp,
Akamai, etc. These include domains including EDA, storage, security,
networking, government (NASA, JPL), embedded systems, business
applications, operating systems, and open source software.  We have
access to Coverity for this course, but there's also a free trial:
\begin{center}
  \url{http://softwareintegrity.coverity.com/FreeTrialWebSite.html}
\end{center}

\paragraph{Mistaken beliefs.} Coverity reported encounters with a number
of mistaken beliefs about how languages work:

``No, the loop will go through once!''
\begin{lstlisting}
  for (i = 1; i < 0; i++) {
    // ... this code is dead ...
  }
\end{lstlisting}

``No, {\tt \&\&} is `or'!''
\begin{lstlisting}
  void *foo(void *p, void *q) {
    if (!p && !q)
      return 0;
  }
\end{lstlisting}

``No, ANSI lets you write 1 past end of the array!''
\begin{lstlisting}
  unsigned p[4]; p[4] = 1;
\end{lstlisting}
(``We'll just have to agree to disagree.'')

\paragraph{Goal.} Coverity aims to find as many serious bugs as possible.
The problem is: what's the definition of a bug? Is there objective truth?
If there is, Coverity doesn't know it.

\squishlist
\item Contradictions: It attempts to find lies by cross-examining; contradictions
indicate errors.
\item Deviance: It assumes programs are mostly-correct and tries to infer
  correct behaviour from that assumption. If 1 person does X, then maybe it's right,
  or maybe that was just a coincidence. But if 1000 people do X and 1 person does Y,
  the 1 person is probably wrong.
\squishend

Crucially: a contradiction constitutes an error, even without knowing
the correct belief.

\paragraph{MUST-beliefs versus MAY-beliefs.} We differentiate between MUST-beliefs
(related to contradictions) and MAY-beliefs (related to deviance).

MUST-beliefs are inferred from acts that imply beliefs about the code.
For instance:
\begin{lstlisting}
  x = *p / z; // MUST: p not null
               // MUST: z != 0
  unlock(l);   // MUST: l acquired
  x++;          // MUST: x not protected by l
\end{lstlisting}

MAY-beliefs, on the other hand, could be coincidental.

\begin{center}
\begin{tabular}{l|l|l|l|l}
\begin{minipage}{5em}
  A();\\
  // ...\\
  B();
\end{minipage} &
\begin{minipage}{5em}
  A();\\
  // ...\\
  B();
\end{minipage} &
\begin{minipage}{5em}
  A();\\
  // ...\\
  B();
\end{minipage} &
\begin{minipage}{5em}
  A();\\
  // ...\\
  B();
\end{minipage} &
\begin{minipage}{20em}
// MAY: A() and B() are paired.
\end{minipage} 
\end{tabular}
\end{center}
We can check them as if they're MUST-beliefs and then rank errors
by belief confidence (more on that later).

\paragraph{MUST-belief examples.} Let's look first at a couple of
MUST-beliefs about null pointers.

\squishlist
\item If I write {\tt *p} in a C program, I'm stating a MUST-belief that
  {\tt p} had better not be {\tt NULL}.
\item If I write the check {\tt p == NULL}, I'm implying two MUST-beliefs:
  1) POST: {\tt p} is {\tt NULL} on true path, not-{\tt NULL} on false path; 2) PRE:
  {\tt p} was unknown before the check.
\squishend

We can cross-check these for three different error types. I leave finding
the actual error to you:
\squishlist
\item check-then-use (79 errors, 26 false positives):
\begin{lstlisting}
  /* linux 2.4.1: drivers/isdn/svmb1/capidrv.c */
  if (!card)
    printk(KERN_ERR, "capidrv-%d: ...", card->contrnr...);
\end{lstlisting}
\item use-then-check (102 errors, 4 false positives):
\begin{lstlisting}
  /* linux 2.4.7: drivers/char/mxser.c */
  struct mxser_struct *info = tty->driver_data;
  unsigned flags;
  if (!tty || !info->xmit_buf)
    return 0;
\end{lstlisting}
\item contradictions/redundant checks (24 errors, 10 false positives:)
\begin{lstlisting}
  /* linux 2.4.7/drivers/video/tdfxfb.c */
  fb_info.regbase_virt = ioremap_nocache(...);
  if (!fb_info.regbase_virt)
    return -ENXIO;
  fb_info.bufbase_virt = ioremap_nocache(...);
  /* [META: meant fb_info.bufbase_virt!] */
  if (!fb_info.regbase_virt) {
    iounmap(fb_info.regbase_virt);
\end{lstlisting}
\squishend

\end{document}

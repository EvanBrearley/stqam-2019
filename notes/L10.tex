\documentclass[11pt]{article}
\usepackage{listings}
\usepackage{tikz}
\usepackage{alltt}
\usepackage{hyperref}
\usepackage{url}
\usepackage{enumitem}
%\usepackage{algorithm2e}
\usetikzlibrary{arrows,automata,shapes}
\tikzstyle{block} = [rectangle, draw, fill=blue!20, 
    text width=5em, text centered, rounded corners, minimum height=2em]
\tikzstyle{bt} = [rectangle, draw, fill=blue!20, 
    text width=1em, text centered, rounded corners, minimum height=2em]

\newtheorem{defn}{Definition}
\newtheorem{crit}{Criterion}
\newcommand{\true}{\mbox{\sf true}}
\newcommand{\false}{\mbox{\sf false}}

\newcommand{\handout}[5]{
  \noindent
  \begin{center}
  \framebox{
    \vbox{
      \hbox to 5.78in { {\bf Software Testing, Quality Assurance and Maintenance } \hfill #2 }
      \vspace{4mm}
      \hbox to 5.78in { {\Large \hfill #5  \hfill} }
      \vspace{2mm}
      \hbox to 5.78in { {\em #3 \hfill #4} }
    }
  }
  \end{center}
  \vspace*{4mm}
}

\newcommand{\lecture}[4]{\handout{#1}{#2}{#3}{#4}{Lecture #1}}
\topmargin 0pt
\advance \topmargin by -\headheight
\advance \topmargin by -\headsep
\textheight 8.9in
\oddsidemargin 0pt
\evensidemargin \oddsidemargin
\marginparwidth 0.5in
\textwidth 6.5in

\parindent 0in
\parskip 1.5ex
%\renewcommand{\baselinestretch}{1.25}

\usepackage{enumitem}

\lstset{ %
language=Java,
basicstyle=\ttfamily,commentstyle=\scriptsize\itshape,showstringspaces=false,breaklines=true,numbers=left}

\usepackage{fontspec}
\setmonofont{Cousine}[Scale=MatchLowercase]

\begin{document}

\lecture{10 --- January 23, 2019}{Winter 2019}{Patrick Lam}{version 1}

\section*{REST APIs}

REST (Representational State Transfer) is an architectural style widely used in web applications.
Architectural styles are SE 464 material, but relevant to Assignment 1, so we'll talk about them briefly here.
[I wouldn't ask questions about REST itself on exams, but one could expect questions about testing REST systems.]

Many applications in the world these days are web apps, and as you know, these apps usually have a
frontend and a backend. Assignment 1 has you testing both the frontend and the backend of my app.
We'll talk about the backend here. For our purposes, it provides an API which allows the client to store data.

Key features of REST:
\begin{itemize}[noitemsep]
\item client-server;
\item stateless; and
\item provides uniform interfaces.
\end{itemize}

Client-server is easy. The client (in our case, the front-end
JavaScript code) sends requests to a server.  The server is
responsible for storing the data. This enables the client to change
somewhat independently of the server. For instance, the server might
decide to change what database it uses behind the scenes.  The client
doesn't need to know about that---separation of concerns.

Statelessness helps systems scale better. What this means is that the
server isn't responsible for remembering anything about client
identities. The server receives requests and processes them
independently. The usual concern is about authentication; requests
need to carry authentication data with them (usually provided by a
separate service). An analogy: because SE is a small program, I try to
remember state in my head about each of you between interactions---at
least your name. Larger programs need students to provide the
necessary state (student ID numbers) with each request; the advisors
would then look you up in a database. You can see how my approach doesn't
scale to larger sets of students.

Uniform interfaces include resources and verbs. Resources are what the
REST service is modifying, e.g. cat pictures. Each resource has a
Uniform Resource Identifier. In our se465-flashcards app, it's just an
ID number. But it could be more complicated, e.g. {\tt /store/1/employee/2}.
Verbs are what to do with the resources. HTTP defines standard
operations, including {\tt POST}, {\tt GET}, and {\tt DELETE}.


Further reading:
\begin{itemize}[noitemsep]
\item Brian Mosigisi. ``A Quick Understanding of REST''. \\
  \url{https://scotch.io/bar-talk/a-quick-understanding-of-rest}
\item Wikipedia is an OK reference here: \url{https://en.wikipedia.org/wiki/Representational_state_transfer}
\end{itemize}

\section*{Testing the REST API}
\paragraph{Sending REST requests.} At the lowest level, we need to be able to send REST requests.
It's easy to send GET requests in the browser by just navigating to a page. You can't quite send other
requests with your bare hands. But you can use the browser developer tools in Firefox and Chrome.
The syntax can be a bit unwieldy; stackoverflow provides the following example\footnote{\url{https://stackoverflow.com/questions/14248296/making-http-requests-using-chrome-developer-tools}}:

{\scriptsize
\begin{lstlisting}
fetch('https://jsonplaceholder.typicode.com/posts', {
  method: 'POST',
  body: JSON.stringify({
    title: 'foo', body: 'bar', userId: 1
  }),
  headers: {
    'Content-type': 'application/json; charset=UTF-8'
  }
})
.then(res => res.json())
.then(console.log)
\end{lstlisting}
}

Or, you can use command-line tools like {\tt cURL} and {\tt wget}. The Firefox dev tools will allow you to
copy a request as cURL and run at the command line, possibly after editing.

But really, you want to script this. REST Assured lets you write Java code like this\footnote{\url{https://semaphoreci.com/community/tutorials/testing-rest-endpoints-using-rest-assured}}:
{\small
  \begin{lstlisting}
     @Test public void makeSureThatGoogleIsUp() {
         given().when().get("http://www.google.com").then().statusCode(200);
     }
  \end{lstlisting}
  }
and run it as a JUnit test.

To check the result, one can compare the status code to what's expected. 200 OK is usually a good HTTP code,
which APIs should return when everything went well. And there's also a result, which one can reason about:
{\small
\begin{lstlisting}
    @Test public void verifyNameStructured() {
        given().when().get("/garage").then().body("name",equalTo("Acme garage"));
    }
\end{lstlisting}
}

You can also construct a POST request using the REST Assured API.

\paragraph{Testing Strategy.} We discussed testing at the tactical level above, which is all you have to do for
Assignment 1. The broader view from this course also includes test suite construction strategies, and
we can talk about various kinds of coverage. You could, for instance, require coverage of the complete API.
Better yet, you could require API coverage and also require that the tests cover the specified behaviour of the API.

\end{document}


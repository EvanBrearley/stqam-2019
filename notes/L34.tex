\documentclass[11pt]{article}
\usepackage{url}
\usepackage{listings}
\usepackage{tikz}
\usepackage{fontspec}
\usepackage{enumitem}
\setmainfont{Latin Modern Roman}
\setmonofont{Cousine}[Scale=MatchLowercase]
\usetikzlibrary{arrows,automata,shapes}
\tikzstyle{block} = [rectangle, draw, fill=blue!20, 
    text width=5em, text centered, rounded corners, minimum height=2em]
\tikzstyle{bt} = [rectangle, draw, fill=blue!20, 
    text width=1em, text centered, rounded corners, minimum height=2em]

\newtheorem{defn}{Definition}
\newtheorem{crit}{Criterion}
\newcommand{\true}{\mbox{\sf true}}
\newcommand{\false}{\mbox{\sf false}}

\newcommand{\handout}[5]{
  \noindent
  \begin{center}
  \framebox{
    \vbox{
      \hbox to 5.78in { {\bf Software Testing, Quality Assurance and Maintenance } \hfill #2 }
      \vspace{4mm}
      \hbox to 5.78in { {\Large \hfill #5  \hfill} }
      \vspace{2mm}
      \hbox to 5.78in { {\em #3 \hfill #4} }
    }
  }
  \end{center}
  \vspace*{4mm}
}

\newcommand{\lecture}[4]{\handout{#1}{#2}{#3}{#4}{Lecture #1}}
\topmargin 0pt
\advance \topmargin by -\headheight
\advance \topmargin by -\headsep
\textheight 8.9in
\oddsidemargin 0pt
\evensidemargin \oddsidemargin
\marginparwidth 0.5in
\textwidth 6.5in

\parindent 0in
\parskip 1.5ex
%\renewcommand{\baselinestretch}{1.25}

\usepackage[listings]{tcolorbox}
\newtcbinputlisting{\codelisting}[3][]{
    extrude left by=1em,
    extrude right by=2em,
    listing file={#3},
    fonttitle=\bfseries,
    listing options={basicstyle=\ttfamily\footnotesize,numbers=left,language=Java,#1},
    listing only,
    hbox,
}
\lstset{ %
language=Java,
basicstyle=\ttfamily,commentstyle=\scriptsize\itshape,showstringspaces=false,breaklines=true,numbers=left}

\definecolor{gray}{rgb}{0.4,0.4,0.4}
\definecolor{darkblue}{rgb}{0.0,0.0,0.6}
\definecolor{cyan}{rgb}{0.0,0.6,0.6}

\lstdefinelanguage{XML}
{
  morestring=[b]",
  morestring=[s]{>}{<},
  morecomment=[s]{<?}{?>},
  stringstyle=\color{black},
  identifierstyle=\color{darkblue},
  keywordstyle=\color{cyan},
  morekeywords={xmlns,version,type}% list your attributes here
}


\begin{document}

\lecture{34 --- March 29, 2017}{Winter 2017}{Patrick Lam}{version 1}

\section*{JML}
The Java Modelling Language allows developers to specify structured
design-by-contract properties for Java code. In particular, you can
specify preconditions and postconditions for methods as well as class
invariants.  These are all in a Java-like expression language.  It is
possible to verify these predicates either statically or dynamically.

\paragraph{Static verification.}
ESC/Java2 is a research tool that reads JML specifications
and enforces them statically using theorem-proving technology. 
Here's some sample code that ESC/Java2 can verify.
{\small \begin{lstlisting}
class Bag {
  int[] a; //@ invariant a != null;
  int   n; //@ invariant 0 <= n && n <= a.length;

  //@ requires n > 0;
  int extractMin() {
    int m = Integer.MAX_VALUE;
    int mindex = 0;
    for (int i = 0; i < n; i++) {
      if (a[i] < m) { mindex = i; m = a[i]; } 
    }
    n--;
    a[mindex] = a[n];
    return m;
  }
}
\end{lstlisting} }
ESC/Java2 reads the Java code and produces verification conditions for the theorem
prover, which in turn uses a backtracking search to assign values to variables.
This tool thus statically guarantees that the code satisfies its specifications.
It's also possible to check these properties at run-time by adding assertions to the code.
% 1. null dereference a[i] < m: add the invariant on a
% 2. array index possibly too large inside loop: add the invariant on n; for loop guarantees
% 3. possible negative array index at a[n]: add the requires n > 0

\paragraph{Dynamic verification.}
It's easier to perform dynamic verification for many JML
specifications.  In the above example, it suffices for the compiler to
insert an assertion that \verb+n > 0+ upon entry to {\tt extractMin},
as well as assumptions of the invariants.  Then upon exit, the
compiler inserts assertions for any {\tt ensures} clauses as well as
the invariants. So we know that the invariant always holds when the
class is not actively working on updating its state.

\end{document}


\documentclass[11pt]{article}
\usepackage{listings}
\usepackage{tikz}
\usepackage{alltt}
\usepackage{hyperref}
\usepackage{url}
\usepackage{enumitem}
%\usepackage{algorithm2e}
\usetikzlibrary{arrows,automata,shapes}
\tikzstyle{block} = [rectangle, draw, fill=blue!20, 
    text width=5em, text centered, rounded corners, minimum height=2em]
\tikzstyle{bt} = [rectangle, draw, fill=blue!20, 
    text width=1em, text centered, rounded corners, minimum height=2em]

\newtheorem{defn}{Definition}
\newtheorem{crit}{Criterion}
\newcommand{\true}{\mbox{\sf true}}
\newcommand{\false}{\mbox{\sf false}}

\newcommand{\handout}[5]{
  \noindent
  \begin{center}
  \framebox{
    \vbox{
      \hbox to 5.78in { {\bf Software Testing, Quality Assurance and Maintenance } \hfill #2 }
      \vspace{4mm}
      \hbox to 5.78in { {\Large \hfill #5  \hfill} }
      \vspace{2mm}
      \hbox to 5.78in { {\em #3 \hfill #4} }
    }
  }
  \end{center}
  \vspace*{4mm}
}

\newcommand{\lecture}[4]{\handout{#1}{#2}{#3}{#4}{Lecture #1}}
\topmargin 0pt
\advance \topmargin by -\headheight
\advance \topmargin by -\headsep
\textheight 8.9in
\oddsidemargin 0pt
\evensidemargin \oddsidemargin
\marginparwidth 0.5in
\textwidth 6.5in

\parindent 0in
\parskip 1.5ex
%\renewcommand{\baselinestretch}{1.25}

%\renewcommand{\baselinestretch}{1.25}
\begin{document}

\lecture{16 --- February 25, 2019}{Winter 2019}{Patrick Lam}{version 1}

Recall that the goals of mutation testing are:
\begin{enumerate}[noitemsep]
\item mimic (and hence test for) typical mistakes;
\item encode knowledge about specific kinds of effective tests in practice, e.g.
statement coverage ($\Delta 4$ from Lecture 15), checking for 0 values ($\Delta 6$).
\end{enumerate}

Reiterating the process for using mutation testing:
\begin{itemize}[noitemsep]
\item \emph{Goal:} kill mutants
\item \emph{Desired Side Effect:} good tests which kill the mutants.
\end{itemize}
These tests will help find faults (we hope). We find these tests by
intuition and analysis.

\section*{Mutation Operators} 

We'll define a number of mutation operators, although precise
definitions are specific to a language of interest. Typical mutation
operators will encode typical programmer mistakes, e.g. by changing
relational operators or variable references; or common testing heuristics, 
e.g. fail on zero. Some mutation operators are better than others.

You can find a more exhaustive list of mutation operators in the PIT documentation:

\begin{center}
\url{http://pitest.org/quickstart/mutators/}
\end{center}

How
many (intraprocedural) mutation operators can you invent for the
following code?

{ \Large
\begin{lstlisting}
int mutationTest(int a, b) { 
  int x = 3 * a, y;
  if (m > n) {
    y = -n;
  }
  else if (!(a > -b)) {
    x = a * b;
  }
  return x;
}
\end{lstlisting}
}

\paragraph{Integration Mutation.} We can go beyond mutating method bodies
by also mutating interfaces between methods, e.g.
\begin{itemize}
\item change calling method by changing actual parameter values;
\item change calling method by changing callee; or
\item change callee by changing inputs and outputs.
\end{itemize}

{
\begin{lstlisting}
class M {
  int f, g;

  void c(int x) {
    foo (x, g);
    bar (3, x);
  }

  int foo(int a, int b) {
    return a + b * f;
  }

  int bar(int a, int b) {
    return a * b;
  }
}
\end{lstlisting}
}

[Absolute value insertion, operator replacement, scalar variable replacement,
  statement replacement with crash statements\ldots]

\paragraph{Mutation for OO Programs.} One can also use some operators specific
to object-oriented programs. Most obviously, one can modify the object on
which field accesses and method calls occur.

{\small
\begin{lstlisting}
class A {
  public int x;
  Object f;
  Square s;

  void m() {
    int x;
    f = new Object();
    this.x = 5;
  }
}

class B extends A {
  int x;
}
\end{lstlisting}
}

\vspace*{-1em}
\paragraph{Exercise.} Come up with a test case to kill each of these types of
mutants.

\begin{itemize}
\item {\bf ABS}: Absolute Value Insertion\\
{\tt x = 3 * a}
$\Longrightarrow$ {\tt x = 3 * abs(a)}, {\tt x = 3 * -abs(a)}, {\tt x = 3 * failOnZero(a)};
\item {\bf ROR}: Relational Operator Replacement\\
{\tt if (m > n)} $\Longrightarrow$ {\tt if (m >= n)}, {\tt if (m < n)}, {\tt if (m <= n)}, {\tt if (m == n)}, {\tt if (m != n)}, {\tt if (false)}, {\tt if (true)}
\item {\bf UOD}: Unary Operator Deletion\\
{\tt if (!(a > -b))} $\Longrightarrow$ {\tt if (a > -b)}, {\tt if (!(a > b))}
\end{itemize}


\vspace*{-1em}
\paragraph{Summary of Syntax-Based Testing.}~\\

\begin{tabular}{l|ll}
& Program-based & Input Space/Fuzzing \\ \hline
Grammar & Programming language & Input languages / XML \\
Summary & Mutates programs / tests integration & Input space testing \\
Use Ground String? & Yes (compare outputs) & Sometimes \\
Use Valid Strings Only? & Yes (mutants must compile) & No \\
Tests & Mutants are not tests & Mutants are tests \\
Killing & Generate tests by killing & Not applicable \\
\end{tabular}

Notes: 
\begin{itemize}[noitemsep]
\item Program-based testing has notion of strong and weak mutants; applied
exhaustively, program-based testing could subsume many other techniques.
\item Sometimes we mutate the grammar, not strings, and get tests from the
mutated grammar.
\end{itemize}

\paragraph{Tool support.} PIT Mutation testing tool: \url{http://pitest.org}. Mutates
your program, reruns your test suite, tells you how it went. You need to distinguish equivalent vs. not-killed.
\end{document}

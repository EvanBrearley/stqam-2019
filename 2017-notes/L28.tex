\documentclass[11pt]{article}
\usepackage{url}
\usepackage{listings}
\usepackage{tikz}
\usepackage{fontspec}
\usepackage{enumitem}
\setmainfont{Latin Modern Roman}
\setmonofont{Cousine}[Scale=MatchLowercase]
\usetikzlibrary{arrows,automata,shapes}
\tikzstyle{block} = [rectangle, draw, fill=blue!20, 
    text width=5em, text centered, rounded corners, minimum height=2em]
\tikzstyle{bt} = [rectangle, draw, fill=blue!20, 
    text width=1em, text centered, rounded corners, minimum height=2em]

\newtheorem{defn}{Definition}
\newtheorem{crit}{Criterion}
\newcommand{\true}{\mbox{\sf true}}
\newcommand{\false}{\mbox{\sf false}}

\newcommand{\handout}[5]{
  \noindent
  \begin{center}
  \framebox{
    \vbox{
      \hbox to 5.78in { {\bf Software Testing, Quality Assurance and Maintenance } \hfill #2 }
      \vspace{4mm}
      \hbox to 5.78in { {\Large \hfill #5  \hfill} }
      \vspace{2mm}
      \hbox to 5.78in { {\em #3 \hfill #4} }
    }
  }
  \end{center}
  \vspace*{4mm}
}

\newcommand{\lecture}[4]{\handout{#1}{#2}{#3}{#4}{Lecture #1}}
\topmargin 0pt
\advance \topmargin by -\headheight
\advance \topmargin by -\headsep
\textheight 8.9in
\oddsidemargin 0pt
\evensidemargin \oddsidemargin
\marginparwidth 0.5in
\textwidth 6.5in

\parindent 0in
\parskip 1.5ex
%\renewcommand{\baselinestretch}{1.25}

\usepackage[listings]{tcolorbox}
\newtcbinputlisting{\codelisting}[3][]{
    extrude left by=1em,
    extrude right by=2em,
    listing file={#3},
    fonttitle=\bfseries,
    listing options={basicstyle=\ttfamily\footnotesize,numbers=left,language=Java,#1},
    listing only,
    hbox,
}
\lstset{ %
language=Java,
basicstyle=\ttfamily,commentstyle=\scriptsize\itshape,showstringspaces=false,breaklines=true,numbers=left}


\begin{document}

\lecture{28 --- March 15, 2017}{Winter 2017}{Patrick Lam}{version 1}

\section*{Code Review}

Code review is a powerful tool for improving code quality. Today's lecture is
based on the following references:

\begin{itemize}
\item course reading on code review (main source):

  \url{http://web.mit.edu/6.031/www/sp17/classes/04-code-review/}
\item how code review works in an MIT course on software construction:

  \url{http://web.mit.edu/6.031/www/sp17/general/code-review.html}
\item Fog Creek code review checklist:

  {\scriptsize \url{https://blog.fogcreek.com/increase-defect-detection-with-our-code-review-checklist-example/} }
\end{itemize}

Our course includes code review as part of Assignment 3. The MIT
offering is more comprehensive and requires students to respond to
code reviews as well.

Code review is a communication-intensive activity. A reviewer needs to 1) read
someone else's code and 2) communicate suggestions to the author of that code.
We sometimes think that communicating with the computer is our primary goal
when programming, but communicating with other people is at least as important
over the long run.

\paragraph{Purpose of code review.} The communication inherent in code review
aims to improve both the code itself as well as the author of the
code.  Good code review can give timely information to developers
about the context in which their code operates,
particularly the project and best-practices uses of the language.

We'll continue with a list of items that you are inspecting when you do a code review.

\paragraph{Formatting.} Consistency in formatting helps avoid preventable
errors. Positioning of \{ \}s isn't something that necessarily has one
right answer. Spaces are probably better than tabs. But the most important thing is
to be self-consistent with yourself and within your project.

\newpage
\section*{Code smell example 1.}

The following code has a number of bad smells:
\begin{lstlisting}
public static int dayOfYear(int month, int dayOfMonth, int year) {
    if (month == 2) {
        dayOfMonth += 31;
    } else if (month == 3) {
        dayOfMonth += 59;
    } else if (month == 4) {
        dayOfMonth += 90;
    } else if (month == 5) {
        dayOfMonth += 31 + 28 + 31 + 30;
    } else if (month == 6) {
        dayOfMonth += 31 + 28 + 31 + 30 + 31;
    } else if (month == 7) {
        dayOfMonth += 31 + 28 + 31 + 30 + 31 + 30;
    }
    // ... through month == 12
    return dayOfMonth;
}
\end{lstlisting}

Let's go through some of the bad smells.

\begin{itemize}
\item {\bf Don't Repeat Yourself}. A few weeks ago in Lecture 23, we talked about code cloning
  and how it's not always bad. Sometimes it is bad, as in the code above. The usual reason for it
  being bad is that fixes in one place may remain unfixed in the other place. (Recall: it wasn't
  always bad when used for forking and templating). For instance, if February actually had 30
  days, you'd need to change a lot of code.
\item {\bf Fail Fast.} In the language of 6.031, we mean that a defect should be caught
  closest to when it's written. Static checks, as performed in compilers, catch defects
  earlier than dynamic checks, which catch defects earlier than letting wrong values percolate
  in the program state. In this particular example, there are no checks ensuring that a
  user had not permuted {\tt month} and {\tt dayOfMonth}.
\item {\bf Avoid Magic Numbers.} The above code is full of magic numbers. Particularly
  magical numbers include the {\tt 59} and {\tt 90} examples, as well as the month lengths
  and the month numbers. Instead, use names like {\tt FEBRUARY} etc. Enums are a good way
  to encode months, and days-of-months should be in an array. The {\tt 59} should really be
  {\tt 31 + 28}, or better yet, \verb!MONTH_LENGTH[JANUARY] + MONTH_LENGTH[FEBRUARY]!.
\item {\bf One Purpose Per Variable.} The specific variable that's
  being re-used in the above example is {\tt dayOfMonth}, but this
  also applies to variables that you might use in your method. Use
  different variables for different purposes. They don't cost
  anything. Best to make method parameters {\tt final} and hence
  non-modifiable.
\end{itemize}

\section*{Comments and code documentation}
Code should, ideally, be self-documenting, with good names for
classes, methods, and variables. Methods should come with
specifications in the form of Javadoc comments, e.g.

{\small
\begin{lstlisting}
/**
 * Compute the hailstone sequence.
 * See http://en.wikipedia.org/wiki/Collatz_conjecture#Statement_of_the_problem
 * @param n starting number of sequence; requires n > 0.
 * @return the hailstone sequence starting at n and ending with 1.
 *         For example, hailstone(3)=[3,10,5,16,8,4,2,1].
 */
public static List<Integer> hailstoneSequence(int n) {
    ...
}
\end{lstlisting}
}

Note how this comment describes what the method does, in one sentence,
provides context, and then describes the parameters and return values.

Also, when you incorporate code from other sources, cite the sources.
For instance, in the A2 {\tt index.html} file:
\begin{lstlisting}
    // adapted from Eli Bendersky's Lexer: http://eli.thegreenplace.net/2013/07/16/hand-written-lexer-in-javascript-compared-to-the-regex-based-ones
    // public domain according to author
    // modifications by Patrick Lam
\end{lstlisting}
This helps, for instance, when the source is later updated, and is the
right thing to do in terms of IP (assuming, of course, that your use of
the software is allowed by its license).

Don't write comments that don't contribute to code understanding.
If it's blatantly obvious from the code, it shouldn't be a comment.
Such comments can mislead and hence do more harm than good.

\end{document}

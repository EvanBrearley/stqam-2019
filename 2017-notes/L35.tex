\documentclass[11pt]{article}
\usepackage{url}
\usepackage{listings}
\usepackage{tikz}
\usepackage{fontspec}
\usepackage{enumitem}
\setmainfont{Latin Modern Roman}
\setmonofont{Cousine}[Scale=MatchLowercase]
\usetikzlibrary{arrows,automata,shapes}
\tikzstyle{block} = [rectangle, draw, fill=blue!20, 
    text width=5em, text centered, rounded corners, minimum height=2em]
\tikzstyle{bt} = [rectangle, draw, fill=blue!20, 
    text width=1em, text centered, rounded corners, minimum height=2em]

\newtheorem{defn}{Definition}
\newtheorem{crit}{Criterion}
\newcommand{\true}{\mbox{\sf true}}
\newcommand{\false}{\mbox{\sf false}}

\newcommand{\handout}[5]{
  \noindent
  \begin{center}
  \framebox{
    \vbox{
      \hbox to 5.78in { {\bf Software Testing, Quality Assurance and Maintenance } \hfill #2 }
      \vspace{4mm}
      \hbox to 5.78in { {\Large \hfill #5  \hfill} }
      \vspace{2mm}
      \hbox to 5.78in { {\em #3 \hfill #4} }
    }
  }
  \end{center}
  \vspace*{4mm}
}

\newcommand{\lecture}[4]{\handout{#1}{#2}{#3}{#4}{Lecture #1}}
\topmargin 0pt
\advance \topmargin by -\headheight
\advance \topmargin by -\headsep
\textheight 8.9in
\oddsidemargin 0pt
\evensidemargin \oddsidemargin
\marginparwidth 0.5in
\textwidth 6.5in

\parindent 0in
\parskip 1.5ex
%\renewcommand{\baselinestretch}{1.25}

\usepackage[listings]{tcolorbox}
\newtcbinputlisting{\codelisting}[3][]{
    extrude left by=1em,
    extrude right by=2em,
    listing file={#3},
    fonttitle=\bfseries,
    listing options={basicstyle=\ttfamily\footnotesize,numbers=left,language=Java,#1},
    listing only,
    hbox,
}
\lstset{ %
language=Java,
basicstyle=\ttfamily,commentstyle=\scriptsize\itshape,showstringspaces=false,breaklines=true,numbers=left}

\definecolor{gray}{rgb}{0.4,0.4,0.4}
\definecolor{darkblue}{rgb}{0.0,0.0,0.6}
\definecolor{cyan}{rgb}{0.0,0.6,0.6}

\lstdefinelanguage{XML}
{
  morestring=[b]",
  morestring=[s]{>}{<},
  morecomment=[s]{<?}{?>},
  stringstyle=\color{black},
  identifierstyle=\color{darkblue},
  keywordstyle=\color{cyan},
  morekeywords={xmlns,version,type}% list your attributes here
}


\begin{document}

\lecture{35 --- March 31, 2017}{Winter 2017}{Patrick Lam}{version 1}

\section*{Dynamic Analyses}
We'll continue talking about generic properties, but this time we'll talk about
dynamic verification of these properties.

\subsection*{Memory errors}
Let's start by talking about leaks and other memory errors.
We saw static detection of leaks with Facebook Infer.
Valgrind's Memcheck tool and Clang's Address Sanitizer detect memory errors
dynamically.

This webpage compares different tools:

\url{https://github.com/google/sanitizers/wiki/AddressSanitizerComparisonOfMemoryTools}


Valgrind's Memcheck detects the following errors:
\begin{itemize}[noitemsep]
\item Illegal reads/writes: Memcheck complains about accesses to memory that
the program should not be accessing (e.g. not returned from a {\tt malloc}, or
below the stack pointer).
\item Reads of uninitialized variables: Memcheck tells you about reads of
memory that has never been initialized, if the program prints out the resulting values.
\item Illegal/wrong frees: Of course, you're not allowed to free memory that
you didn't get from a memory allocation function, so Memcheck tells you.
It also tells you when you use {\tt free()} on something you got from {\tt new[]}.
\item Overlapping source/destination for memory moves.
\item Fishy argument values: You probably don't mean to request either -3
bytes or more than $2^{63}$ bytes.
\item Memory leaks: Memcheck tells you when you have memory that didn't get freed
but that you no longer have any pointers to.
\end{itemize}
You will pay a significant performance penalty when using Valgrind; Memcheck typically
comes with a 10$\times$--50$\times$ slowdown. This is usable for bug diagnosis but
obviously not for production.

Here is more information on what Valgrind can do. It can do more than just
Memcheck as well.
\begin{center}
\url{http://maintainablecode.logdown.com/posts/245425-valgrind-is-not-a-leak-checker}
\end{center}

The AddressSanitizer tool, supported by both clang and gcc,
also detects memory errors. These errors are similar to those that
Valgrind can detect. The technology is different, but the goals are
similar. Because it uses different implementation technology, it runs
much more quickly than Valgrind, with a reported typical slowdown of
2$\times$.

\pagebreak
Find some information about AddressSanitizer here:
\vspace*{-1em}
\begin{center}
\url{http://btorpey.github.io/blog/2014/03/27/using-clangs-address-sanitizer/}
\end{center}
\vspace*{-1em}

AddressSanitizer finds the following errors:
\vspace*{-1em}
\begin{itemize}[noitemsep]
\item    Out-of-bounds accesses to heap, stack and globals
\item    Use-after-free
\item    Use-after-return
\item    Use-after-scope
\item    Double-free, invalid free
\item    Memory leaks (experimental)
\end{itemize}
\vspace*{-1em}

Note that Valgrind may return false positives, while AddressSanitizer has a design goal 
of returning no false positives. In fact, ASan aborts the program whenever it encounters
a memory problem. You're supposed to fix it before proceeding. Of course, ASan might miss 
some problems.

\paragraph{Implementation Techniques.}
Valgrind and ASan use different techniques, hence yield different results.
Valgrind emulates a CPU and checks, at every memory access, whether that
access is legitimate or not. ASan, on the other hand, rewrites relevant memory accesses
at compile-time to call a checking library. It turns out that calling a library is cheaper
than emulating the CPU.

Conceptually, ASan maintains shadow memory---
metadata about where the program is allowed to access (or not).
Then, it replaces the {\tt malloc()} and {\tt free()}
calls with its own versions; these versions update shadow memory and indicate
that allocated memory is OK to access, unallocated memory not OK. 
Finally, every memory access in the program gets replaced with a checked access:
\begin{lstlisting}
if (IsPoisoned(address)) {
  ReportError(address, kAccessSize, kIsWrite);
}
*address = ...;  // or: ... = *address;
\end{lstlisting}
This is not so different from what Java does to check array accesses, for instance,
but applies much more generally, to every memory access in the program.

More details about ASan:
\vspace*{-1em}
\begin{center}
\url{https://github.com/google/sanitizers/wiki/AddressSanitizerAlgorithm}
\end{center}
Both of these tools are open-source. The source code is the ultimate description of the tools.

\subsection*{Race detectors}
Valgrind also has a race detection tool, Helgrind. This tool dynamically detects
memory that is concurrently accessed by two threads. Such accesses are fine as long
as they are controlled by a lock. At runtime, Helgrind knows which locks are held.
If the program does not hold enough locks, then Helgrind flags a memory error.

\hspace*{.2\textwidth}\begin{minipage}{.4\textwidth}
\begin{lstlisting}
  acquire(lock1);
  read(x);
  release(lock1);
\end{lstlisting}
\end{minipage}\
\begin{minipage}{.4\textwidth}
\begin{lstlisting}
  acquire(lock2);
  write(x); // different lock!
  release(lock2);
\end{lstlisting}
\end{minipage}

%test generation tools?
%security
%fuzzing was a dynamic analysis

%dynamic analysis: concurrency---race detectors
%what else as a dynamic analysis?

\end{document}
